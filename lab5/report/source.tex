\section{Appendix A: Answers to inline questions}
\subsection{Question 1}
We produce output every clock cycle. At 100 MHz we could produce
$\frac{\frac{100 \cdot 10^3}{1}}{48} \approx 2083$ streams.
\subsection{Question 2}
Our design can process 1024 input streams. It does however not run at 100 MHz, we achieve this number of streams by producing output every clock cycle. The downside of this is that the \texttt{ngrid} server does not compile design not running at at least 100 MHz, so we could not test in on the Xilinx board.
\subsection{Question 3}
Section ~\ref{sec:analysis} contains an analysis of the input and output samples. There is one stream that represents an audio file, the others are zero. We have checked that the output audio is correct and has a sample frequency of 48 KHz. In our design we make no distinction between the streams, they are processed all in the same way. We do not use the number of the stream that contains actual data. Therefore it is reasonable to assume that all streams are correctly processed and have an output sample frequency of 48 KHz. 
\section{Appendix B: Verilog source code}
This appendix includes Verilog source code for the \texttt{filter.v} file in the ISE project.

\begin{verbatim}
`timescale 1ns / 1ps

module filter
    #(parameter DWIDTH = 16,
        parameter DDWIDTH = 2*DWIDTH,
        parameter L = 160,
        parameter L_LOG = 8,
        parameter M = 147,
        parameter M_LOG = 8,
        parameter CWIDTH = 4*L,
        parameter NR_STREAMS = 1024,
        parameter NR_STREAMS_LOG = 10)
    (input clk,
        input rst,
        output req_in,
        input ack_in,
        input [0:DWIDTH-1] data_in,
        output req_out,
        input ack_out,
        output [0:DWIDTH-1] data_out);

    // Output request register
    reg req_out_buf;
    assign req_out = req_out_buf;

    // Input request register
    reg req_in_buf;
    assign req_in = req_in_buf;

    reg [0:NR_STREAMS_LOG-1] stream;

    // state counter. l = nM mod L (calculated efficiently using 
    // conditionals and addition/subtraction)
    reg [0:L_LOG-1] l;
    // Delayed state counter, to compensate for the fact that 
    // I/O and computation happen simultaneously now.
    reg [0:L_LOG-1] m;
    // The last 4 input samples used in the FIR (excluding 
    // the newest, which will be in data_in)
    reg signed [0:DWIDTH-1] in0 [0:NR_STREAMS-1];
    reg signed [0:DWIDTH-1] in1 [0:NR_STREAMS-1];
    reg signed [0:DWIDTH-1] in2 [0:NR_STREAMS-1];
    reg signed [0:DWIDTH-1] in3 [0:NR_STREAMS-1];

    // The FIR coefficients (lots of them)
    // Naively, we'd need [0:4L-1], but since the coefficients are
    // symmetric around h[2L], we can just reuse h[2L-1:1] for // h[2L+1:4L-1]
    reg signed [0:DWIDTH-1] h [0:2*L];

    // Accumulator (lower bits assigned to output port directly)
    reg signed [0:DDWIDTH-1] sum;
    assign data_out = sum >> 15;


    initial
    begin
        // Initialize the ROM with coefficients from file
        $readmemh("coefficients.txt", h);
    end


    always @(posedge clk)
    begin
        // Reset => initialize
        if (rst)
        begin
            req_in_buf <= 0;
            req_out_buf <= 0;
            stream <= 0;
            l <= 0;
        end
        // !Reset => run
        else
        begin

            // Read handshake complete
            if (req_in && ack_in)
            begin
                in0[stream] <= data_in;
                in1[stream] <= in0[stream];
                in2[stream] <= in1[stream];
                in3[stream] <= in2[stream];

                //sum <= (data_in >> 1) | (data_in & 32768);
                req_out_buf <= 1;
            end

               //Read handshake is pending then stop producing output
            if (req_in && !ack_in)
            begin
                req_out_buf <= 0;
            end

            // Write handshake complete
            if (req_out && ack_out)
            begin
                if( stream == 0 )
                begin
                    // l <= (l + M) mod L, implemented with conditionals
                    // No-overflow case
                    if( l < L - M )
                    begin
                        l <= l + M;
                        req_in_buf <= 0;
                    end
                    // Overflow case. Overflow also means we need a new input!
                    else
                    begin
                        l <= l - (L - M);
                        req_in_buf <= 1;
                    end

                    m <= l;
                end

                sum <= in0[stream] * h[m] + in1[stream] * h[m+L] + in2[stream]
                          * h[L*2-m] + in3[stream] * h[L-m];

                stream <= (stream + 1) & (NR_STREAMS-1);
            end

            //Write handshake is pending then stop acquiring output.
            if (req_out && !ack_out)
            begin
                req_in_buf <= 0;
            end

            // Idle state
            if (!req_in && !ack_in && !req_out && !ack_out)
            begin
                req_in_buf <= 1;
            end

        end
    end

endmodule

\end{verbatim}
