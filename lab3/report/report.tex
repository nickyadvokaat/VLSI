\documentclass[a4paper,twoside,11pt]{article}
\usepackage{a4wide,graphicx,subfigure,fancyhdr,amsmath,amssymb,algpseudocode,enumerate,hyperref, float,placeins}
\usepackage[english]{babel}
\numberwithin{equation}{section}

%----------------------- Macros and Definitions --------------------------

\setlength\headheight{20pt}
\addtolength\topmargin{-10pt}
\addtolength\footskip{20pt}

\newcommand{\N}{\mathbb{N}}
\newcommand{\ch}{\mathcal{CH}}

\fancypagestyle{plain}{%
\fancyhf{}
\fancyhead[LO,RE]{\sffamily\bfseries\large}
\fancyhead[RO,LE]{\sffamily\bfseries\large }
\fancyfoot[LO,RE]{\sffamily\bfseries\large }
\fancyfoot[RO,LE]{\sffamily\bfseries\thepage}
\renewcommand{\headrulewidth}{0pt}
\renewcommand{\footrulewidth}{0pt}
}

\pagestyle{fancy}
\fancyhf{}
\fancyhead[RO,LE]{\sffamily\bfseries\large 2IN35}
\fancyhead[LO,RE]{\sffamily\bfseries\large Lab 3 }
\fancyfoot[LO,RE]{\sffamily\bfseries\large }
\fancyfoot[RO,LE]{\sffamily\bfseries\thepage}
\renewcommand{\headrulewidth}{1pt}
\renewcommand{\footrulewidth}{0pt}
\newcommand{\name}{VLSI Programming}

%-------------------------------- Title ----------------------------------

\title{\vspace{-\baselineskip}\sffamily\bfseries VLSI Lab 3}

\author{
Nicky Advokaat - 0740567 - {\tt n.advokaat@student.tue.nl} \\
Marcel  Moreaux - 0499480 - {\tt  m.l.moreaux@student.tue.nl}\\
}

\date{4\textsuperscript{rd} quartile, 2014}

%--------------------------------- Text ----------------------------------

\begin{document}
\maketitle
\thispagestyle{empty}
\begin{abstract}
This report contains solutions for the problems described in Assignment L3 for the course VLSI Programming. We will transform a parallel FIR filter to a sequential one to reduce hardware utilization. We will also create a strength reduced version of this filter.
\end{abstract}

\tableofcontents

\newpage

\section{Problem Specification and Requirements}
We need to implement an upscaler that can process $n$ streams at once, where all streams have the same sample rate and the same upscaling factor. This means we can reuse most of our code from L4 in which we created a single stream upscaler. The calculations performed for each stream are the same as for a single stream, so we just need to store more inputs and interleave the filtering. The coefficients and the way they are stored are equal can be copied from assignment L4 as well. 
The upscaler has the following requirements:
\begin{itemize}
\item The system must run at 100 MHz.
\item The system can handle at least 128 streams, but preferably more. We will incrementally test the filter on the number of input streams.
\item All streams are correctly upscaled from 44.1 kHZ to 48 kHZ and outputted in the correct order.
\end{itemize}

\section{Solution}
In this section we describe the key ideas behind our design, and the decisions we made during the design process.
\subsection{Sequential Implementation}
To implement a sequential filter we had to solve a number of problems. In between each data input, we have to do 32 multiply accumulate operations. Therefore we have to buffer the latest 32 input values in an array \texttt{data}, in which the 0th index contains $x(i)$ and the last value is $x(i -31)$. This can be seen at the top of figure ~\ref{fig:architecture}. So in each step, the new data value is supplied to \texttt{data}[0], and for all $1 \leq i \leq 31$ the value of \texttt{data}[$i$-1] is copied to \texttt{data}[$i$]. An alternative solution would be to create a ring buffer, in which we supply the input value to the $(n\bmod 32)$-th slot of \texttt{data}. We decided not to use this approach, because it would require an extra register to point to the latest variable, and an extra multiplexer to index the input wires of the array.\\
After the $n^{th}$ input is supplied we need to calculate $y(n)=\Sigma_{0\leq i \leq 31}(h(i)\cdot x(n-i))$. In our design this corresponds to $\Sigma_{0 \leq i \leq 31}(data[i] * h\_in[i])$.  

\begin{figure}
\includegraphics[width=0.9\textwidth]{images/architecture.png}
\caption{Architecture diagram of the sequential FIR filter.}
\label{fig:architecture}
\end{figure}


\section{Results}
In this section we report statistics and resource utilization of our design, and analyze the resulting wave forms.
\subsection{Sequential Implementation}
\subsubsection{Resources}
todo
\subsubsection{Properties}
The sample frequency of the filter is as follows:
\begin{itemize}
\item
Synthesis report
\begin{itemize}
\item Minimum period: 7.351ns
\item Maximum Frequency: 136.037MHz
\end{itemize}
\item
Post-PAR static timing report
\begin{itemize}
\item  Minimum period:   9.137ns  
\item Maximum frequency: 109.445MHz
\end{itemize}
\end{itemize}
The sample frequency is lower in the static timing report, because it takes into account how the circuit will actually be laid out on the FPGA. The important thing here is that the sample frequency meets the 100 MHz requirement, which means this filter is fast enough to process the audio file using only one multiplier.
\subsection{Strength Reduced}
\subsubsection{Resources}
todo
\subsubsection{Properties}
todo
\subsection{Analysis of filter output}
Here we will analyze the effect of the filter on the input signal. This section corresponds to both the sequential and the strength reduced filter, since they implement the same function, a low pass FIR filter.\\
The effect of the filter is clearly shown in figure  ~\ref{fig:diff}, where the original signal is plotted together with the filtered signal and the difference between the two signals. It is clear to see that the high frequency harmonics in the original signal have been removed in the filtered signal, because it is much smoother. This is also demonstrated by in the difference signal, which only shows high harmonics. To calculate the difference, we used a scaling factor because the filtered signal is weaker than the original signal. This is caused by the filter coefficients which are:
\begin{gather*}
14,15,16,17,17,16,15,14
\end{gather*}
The sum of these coefficients is $124$, which gives us a scaling factor of $\frac{256}{124}$. We also shifted the filtered signal by 4, half of the number of coefficients, to have the signals line up better. The figure also shows the start up noise of the filter; the first 32 values are incorrect.

\begin{figure}
\begin{center}
\includegraphics[width=0.9\textwidth]{images/diff.png}
\caption{Part the original signal (red), the filtered signal (blue), and the signal \mbox{(original - $\frac{256}{124}$ * filtered signal)} (green).}
\label{fig:diff}
\end{center}
\end{figure}

Another visualization of the effect of the filter is shown in figure  ~\ref{fig:spectrum} in which the signals are plotted in the frequency domain. The high frequency harmonics are clearly missing in the filtered signal, the stop frequency is around 4000 Hz. 

\begin{figure}
\begin{center}
\includegraphics[width=0.7\textwidth]{images/spectrum_input.png}
\includegraphics[width=0.7\textwidth]{images/spectrum_output.png}
\caption{Plot in the frequency domain of the original signal (top) and the signal after filtering (bottom).}
\label{fig:spectrum}
\end{center}
\end{figure}


\FloatBarrier


\section{Appendix A: Answers to inline questions}
\subsection{Question 1}
\begin{equation*}
y[n] = z[Mn]
\end{equation*}
\begin{equation*}
z[n] = \sum_{0\leq j < 4L} h[j] \cdot q[n-j]
\end{equation*}
\begin{equation*}
q[n] = \begin{cases}
x[n \text{ div } L]&\text{if } n \bmod L = 0\\
0&\text{otherwise}
\end{cases}
\end{equation*}
\begin{equation*}
y[n] = \sum_{0\leq j<4L} h[j] \cdot q[nM-j]
\end{equation*}
\begin{equation*}
y[n] = \begin{cases}
\sum_{0 \leq j < 4L} h[j] \cdot q[(nM-j) \text{ div } L]&\text{if } (nM - j) \bmod L = 0\\
\sum_{0 \leq j < 4L} h[j] \cdot 0 &\text{otherwise}
\end{cases}
\end{equation*}
The \emph{otherwise} case is always 0 so doesn't contribute to the sum.
We can continue with just:
\begin{equation*}
y[n] = \sum_{0\leq j<4L} h[j] \cdot q[(nM-j) \text{ div } L])  \phantom{xxx}\text{ if } (nM-j) \bmod L = 0
\end{equation*}
$(nM-j) \bmod L = 0$  happens at:
\begin{itemize}
\item $  j = nM \bmod L$
\item $  j = nM \bmod L + L$
\item $  j = nM \bmod L + 2L$
\item $  j = nM \bmod L + 3L$
\end{itemize}
So four times in every summation. If we let $j$ run from $0$ to $4$ (excl),
we get $h[nM \bmod L + jL]$.
\begin{equation*}
y[n] = \sum_{0\leq j<4} h[nM \bmod L + jL] \cdot q[(nM-j \cdot L) \text{ div } L]
\end{equation*}
\begin{equation*}
y[n] = \sum_{0\leq j<4} h[nM \bmod L + jL] \cdot q[(nM) \text{ div } L - j]
\end{equation*}
Which is the equation from the assignment. 

\subsection{Question 2}
\subsection{Question 3}
\section{Appendix B: Verilog source code}
This appendix includes Verilog source code for the \texttt{filter.v} file in the ISE project.

\begin{verbatim}
\end{verbatim}


\end{document}
