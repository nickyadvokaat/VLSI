\section{Problem Specification and Requirements}
The goal of this assignment is to construct an upscaler that increases the sample rate of an input signal from 44.1 KHZ to 48 KHZ, which gives the upscaling ratio $\frac{160}{147}$.  This is done by first upscaling by L=160, then applying a filter, and the downscaling by M=147.  The coefficients for the filter are constructed from the \texttt{lanczos} function, which is a finite window version of the \texttt{sinc} function. We will have to generate these coefficients ourselves, and store them in ROM memory. There several alternatives to construct the upscaler, these are discussed in section ~\ref{sec:Q3}. The upscaler must have a clock frequency of at least 100 MHz. We have a choice whether to optimize the upscaler for minimum resource utilization, or for maximum throughput. After we have designed and implemented the upscaler we will test it by analyzing the input and output signals.