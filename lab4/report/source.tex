\section{Appendix A: Answers to inline questions}
\subsection{Question 1}
\begin{equation*}
y[n] = z[Mn]
\end{equation*}
\begin{equation*}
z[n] = \sum_{0\leq j < 4L} h[j] \cdot q[n-j]
\end{equation*}
\begin{equation*}
q[n] = \begin{cases}
x[n \text{ div } L]&\text{if } n \bmod L = 0\\
0&\text{otherwise}
\end{cases}
\end{equation*}
\begin{equation*}
y[n] = \sum_{0\leq j<4L} h[j] \cdot q[nM-j]
\end{equation*}
\begin{equation*}
y[n] = \begin{cases}
\sum_{0 \leq j < 4L} h[j] \cdot q[(nM-j) \text{ div } L]&\text{if } (nM - j) \bmod L = 0\\
\sum_{0 \leq j < 4L} h[j] \cdot 0 &\text{otherwise}
\end{cases}
\end{equation*}
The \emph{otherwise} case is always 0 so doesn't contribute to the sum.
We can continue with just:
\begin{equation*}
y[n] = \sum_{0\leq j<4L} h[j] \cdot q[(nM-j) \text{ div } L])  \phantom{xxx}\text{ if } (nM-j) \bmod L = 0
\end{equation*}
$(nM-j) \bmod L = 0$  happens at:
\begin{itemize}
\item $  j = nM \bmod L$
\item $  j = nM \bmod L + L$
\item $  j = nM \bmod L + 2L$
\item $  j = nM \bmod L + 3L$
\end{itemize}
So four times in every summation. If we let $j$ run from $0$ to $4$ (excl),
we get $h[nM \bmod L + jL]$.
\begin{equation*}
y[n] = \sum_{0\leq j<4} h[nM \bmod L + jL] \cdot q[(nM-j \cdot L) \text{ div } L]
\end{equation*}
\begin{equation*}
y[n] = \sum_{0\leq j<4} h[nM \bmod L + jL] \cdot q[(nM) \text{ div } L - j]
\end{equation*}
Which is the equation from the assignment. 

\subsection{Question 2}
The period is $L$.

\subsection{Question 3}
We assume all values for $h[]$ are precomputed.
\begin{itemize}
\item Naive scaler directly
\begin{itemize}
\item Multiplications. The number of coefficients is $4L$. For each sample we need the multiply for each coefficient. The frequency of the FIR is $147$ times the output sample frequency. So the number of multiplications per output is $147 \cdot 640 = 94080$.
\item Coefficients. $4L = 4 \cdot 64 = 640$.
\item Highest sample rate. $94080 \cdot 48 \text{ kHz } \approx 4.516  \text{ GHz }$ assuming sequential FIR implementation.
\end{itemize}
\item Naive scaler composed
\begin{itemize}
\item Multiplications. We have $\frac{M_2}{L_2} \cdot M_1 \cdot (4 \cdot L_1) + M_2 \cdot (4 \cdot L_2) = \frac{14}{15} \cdot 63 \cdot (4 \cdot 64) + 14 \cdot (4 \cdot 15) = 15892.8$ multiplications per output. 
\item Coefficients. $4 \cdot L_1 + 4 \cdot L_2 = 4 \cdot 64 + 4\cdot 15 = 316$.
\item Highest sample rate. The second sub filter has the highest sample rate at $(4 \cdot 64) \cdot 63 \cdot \frac{14}{15} \cdot 48 \text{ kHz } \approx 722\text{ MHz }$, using sequential FIR.
\end{itemize}
\item Equation scaler directly
\begin{itemize}
\item Multiplications. In the direct formula the term $jL$ can be precomputed since $0\leq j<4$. The term $nM$ does not need a multiplier because $n$ is incremented by one each time, so we can just add $M$ to a stored value. For the same reason the \emph{mod} and \emph{div} do not need multipliers. In the end this only leave the multiplication for $h[]\cdot x[]$. This is done for each $j$ over the summation, so we need 4 multiplications. 
\item Coefficients. We need $4L = 640$ coefficients.
\item Highest sample rate. The filter runs at 4 times the output sample rate, $4 \cdot 48 \text{ kHZ } = 192$ kHz.
\end{itemize}
\item Equation scaler composed
\begin{itemize}
\item Multiplications. Using the same arguments as with the direct equation scaler we get $4 + 4 \cdot 14/15 \approx 7.7$ multiplications per output.
\item Coefficients. $4 \cdot 64 + 4\cdot 15 = 316$.
\item Highest sample rate. The second FIR runs at 4 times the output sample rate, $4 \cdot 48 \text{ kHZ } = 192$ kHz.
\end{itemize}
\end{itemize}
\section{Appendix B: Verilog source code}
This appendix includes Verilog source code for the \texttt{filter.v} file in the ISE project.

\begin{verbatim}
\end{verbatim}
